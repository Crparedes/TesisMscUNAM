%% For tips on how to write a great abstract, have a look at
%%	-	https://www.cdc.gov/stdconference/2018/How-to-Write-an-Abstract_v4.pdf (presentation, start here)
%%	-	https://users.ece.cmu.edu/~koopman/essays/abstract.html
%%	-	https://search.proquest.com/docview/1417403858
%%  - 	https://www.sciencedirect.com/science/article/pii/S037837821830402X

\begin{resumen}%\addcontentsline{toc}{chapter}{Resumen}
\selectlanguage{spanish}
Este trabajo de tesis presenta una metodología para extraer ion litio que se encuentra a baja concentración en medios acuosos, usando una membrana polimérica de inclusión que incorpora los extractactes comerciales LIX-54-100 y Cyanex~923, en un soporte polimérico de triacetato de celulosa. Se logró la extracción de ion litio empleando, como matriz de extracción, disoluciones ideales, mezclas sintéticas donde el ion litio está presente en conjunto con otros cationes, una matriz de agua de mar sintética simplificada, y muestras de agua de mar natural. El sistema presenta buena selectividad por el el ion litio frente los cationes potasio y sodio. Es posible concentrar el ion litio usando la metodología propuesta, y la membrana es medianamente estable durante los primeros diez ciclos de transporte. La optimización del sistema se hizo por medio de diseños experimentales factoriales fraccionados de dos niveles y siguiendo el algoritmo de optimización simplex de paso variable.

Los perfiles de transporte fueron ajustados a una nueva ecuación empírica que se propone en este trabajo. El modelo presenta un excelente ajuste con los datos experimentales, y sus parámetros ajustables se pueden relacionar con la velocidad y con la eficiencia del proceso de transporte. Esto permitió que dichos parámetros pudieran ser usados exitosamente como variables respuesta en los diseños experimentales desarrollados.

En el desarrollo del proyecto se escribió un paquete para el programa de computación científica y representaciones gráficas \R. El propósito del paquete \verb|transmem| es facilitar el tratamiento sistemático y reproducible de los datos generados en experimentos de transporte a través de membranas, con el fin de obtener parámetros de desempeño del sistema y de producir representaciones gráficas de alta calidad. El paquete está disponible en el repositorio oficial de \R, y actualmente se encuentra en proceso de adaptación a una interfaz gráfica interactiva (tipo aplicación web), que permite el aprovechamiento de sus funcionalidades por parte de usuarios que no se encuentran familiarizados con el lenguaje de programación \R.
%Los resultados más relevantes del presente trabajo fueron recopilados en un artículo aceptado para su publicación en la revista \textit{Desalination}.\footnote{ISSN: 0011-9164}

\end{resumen}
%\begin{abstract}\addcontentsline{toc}{chapter}{Abstract}
%\textbf{This is the abstract.}
%\end{abstract}