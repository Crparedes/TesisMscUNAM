\begin{acknowledgements}
Agradezco a la espléndida Universidad Nacional Autónoma de México que me acogió con calidez y me hizo sentir siempre como parte de su hermosa comunidad. Agradezco a mi tutor, el Dr. Eduardo Rodríguez de San Miguel, por ser el arquitecto del proyecto de investigación y por compartirme conocimientos que atesoraré toda la vida. Agradezco a los respetados miembros de mi Jurado Evaluador por sus oportunas observaciones y correcciones que fueron fundamentales para mejorar la calidad de este manuscrito. También agradezco al Consejo Nacional de Ciencia y Tecnología, por administrar y compartirme parte de los recursos de los contribuyentes del hermoso pueblo Mexicano, quienes de manera generosa y desinteresada financiaron mi estancia es su lindo país lleno de arte, historia, magia y color, bajo el número de CVU 918468. Llevaré siempre a México y a su ciudadanía incrustados en mi corazón. 

Del personal académico de la UNAM agradezco a la Dra. Josefina de Gyves por los consejos y el apoyo que me otorgó durante la maestría. Aprecio los servicios técnicos de QFB. Guadalupe Espejel y Qca. Nadia Munguía. Valoro la compañía y el interés de mis atentos amigos compañeros de laboratorio, y la de mi amigo Corzo, quien tuvo la fabulosa idea de considerar a la UNAM como casa de estudios para adelantar mis estudios de posgrado.

Agradezco a mi amada Alma Mater, la Universidad Nacional de Colombia, y a las buenas personas que me crucé en mi paso por esa Institución. En especial agradezco a los Profesores Jesús Agreda y Eliana Castillo, por creer en mí y por su influencia enorme en mi forma de ver el mundo que me rodea.

Debo agradecer a mis padres Albita Cardona y Víctor Paredes, quienes con su ejemplo, su apoyo y su amor, me han motivado para seguir trabajando en ser cada vez una mejor persona. También les debo a ellos la recolección de las muestras de agua de mar que fueron usadas en la etapa final de este trabajo. A mis hermanos Vickita y Branditon, que siempre estuvieron presentes en mis felicidades y mis penurias. Me considero muy afortunado de tenerlos en mi vida. 

Finalmente quiero hacer una mención muy especial a Alexandra Elbakyan por su incansable esfuerzo en remover las barreras del acceso al conocimiento. %, y finalmente, al Maestro de Capilla Juan Sebastian Bach, por su obra sublime que siempre me logró reconfortar en los momentos difíciles y que, a mi modo de ver, siempre será el regalo más hermoso que pudo alguien haber hecho a la humanidad.
\end{acknowledgements}