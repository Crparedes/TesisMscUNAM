\chapter{Introducción}

\section{Descripción del problema}\label{sec:descrip} % why is this a non trivial problem
La civilización moderna hace uso de vastas cantidades de energía para sostener distintas actividades que considera primordiales para su desarrollo. Los homínidos empezaron a dominar el fuego hace aproximadamente dos millones de años, y esto representó un punto de inflexión importante en el camino para convertirse en lo que somos nosotros ahora \citep{Gowlett2016}. Desde entonces, la combustión ha representado la fuente energética por excelencia para impulsar diversos procesos. Los recursos fósiles son el combustible más común, pero su uso presenta varias desventajas que conciernen principalmente al impacto ambiental negativo que generan y al hecho de que su disponibilidad es limitada. Por otro lado, las fuentes energéticas renovables son más limpias desde un punto de vista ambiental y en algunos casos pueden considerarse de disponibilidad ilimitada para todos los efectos prácticos. El principal problema que reside en su aprovechamiento es que su disponibilidad no es constante en el tiempo por lo que debe almacenarse eficientemente. Una forma práctica de almacenar energía es en forma de energía eléctrica. Los dispositivos de almacenamiento de energía eléctrica tienen un papel protagónico en la revolución ambiental y energética que se experimenta en el mundo actualmente. En particular, las baterías de ion de litio (LIB) han probado ser una excelente alternativa para almacenar energía eléctrica \citep{Chen2020}, y su uso se ha venido expandiendo en los últimos años.

Algunos autores consideran que la gran ganancia del mercado por parte de las LIBs puede repercutir en escasez y desabastecimiento de litio cuando la demanda supere a la oferta, en el futuro, si no se toman medidas para evitarlo \citep{SVERDRUP2016, VIKSTROM2013, Benson2017, Chagnes2015}. Actualmente, este elemento es extraído a escala industrial a partir de minerales y salmueras ubicados en algunas regiones del mundo. Igual que los combustibles fósiles, los recursos de los que el litio puede ser extraído de una forma económicamente viable usando la tecnología disponible hoy en día, son limitados. Varias posibles fuentes naturales de litio no son explotadas actualmente, principalmente porque el litio se encuentra demasiado diluido, o porque la presencia de especies interferentes que dificultan su recobro es muy alta. El litio es considerado por muchos como un elemento crítico, es decir, un elemento que presenta riesgos en su cadena de suministro \citep{Zubi2018}. Un gran número de grupos de investigación en todo el mundo trabaja muy activamente en el desarrollo y la adaptación de técnicas de separación novedosas para la extracción de litio desde distintas fuentes. 

El agua de los mares constituye la reserva de litio conocida más grande del planeta. El ion litio se encuentra muy diluido en esta matriz (cerca de 200~$mg~m^{-3}$), pero debido al gran volumen del agua de mar, esta podría actuar como una fuente casi ilimitada de este recurso \citep{Yang2018}. La concentración molar de distintos elementos interferentes presentes en el agua de mar es de hasta tres y cuatro órdenes de magnitud mayor a la del ion litio, y esto propone un reto adicional a su extracción desde esta matriz \citep{LI2019117317}. El desarrollo de una metodología apropiada para la extracción selectiva de ion litio a partir de agua de mar podría asegurar su suministro para muchas décadas por venir. Sin embargo, esta matriz puede ser la fuente líquida a partir de la cual la extracción de ion litio es la más difícil de lograr debido a su muy baja concentración y a la muy alta concentración de las especies interferentes.

La propuesta en el presente trabajo radica en el uso de una membrana polimérica de inclusión (PIM\acused{PIM}) para el recobro de ion litio disponible en el agua de mar. Las \ac{PIM}s han sido ampliamente usadas para la extracción selectiva de un gran número de iones metálicos y moléculas orgánicas pequeñas \citep{Nghiem2006}. Algunos de los reportes involucran el recobro de cationes metálicos a partir de agua de mar en donde se encuentran a bajas concentraciones \citep{Pont2008, Scindia2005}. Al inicio del presente proyecto, en agosto del 2018, no existían reportes de recobro o transporte de ion litio usando una PIM \citep{Cai2019}. Actualmente, hasta donde sabemos, esta técnica no ha sido aplicada a la extracción de ion litio a partir de una muestra real.

\section{Objetivos}
El objetivo general del presente trabajo es \textbf{proponer las condiciones adecuadas para extraer y concentrar selectivamente el ion litio presente a bajas concentraciones en medios acuosos, haciendo uso de una membrana polimérica de inclusión, y aplicar la metodología desarrollada a la recuperación de ion litio a partir de una muestra real de agua de mar}.

Los objetivos específicos que se consideran apropiados para alcanzar el objetivo general son:
\begin{itemize}
    \item Proponer los extractantes más adecuados para la \ac{PIM} mediante experimentos de extracción líquido-sólido, considerando la amplia información disponible en la literatura concerniente al recobro de ion litio a partir de fuentes acuosas.
    \item Optimizar las condiciones del sistema (composición de la membrana y de las disoluciones de alimentación y recuperación) para extraer selectivamente ion litio usando celdas de permeación.
    \item Adecuar las técnicas de medición apropiadas para determinar las magnitudes de interés que permiten monitorear los procesos de transporte.
    \item Determinar los parámetros de desempeño del sistema optimizado:
    \begin{itemize}
        \item Coeficiente de permeabilidad del ion litio en la PIM
        \item Selectividad del sistema frente a otros cationes metálicos
        \item Capacidad de reuso de la membrana
    \end{itemize}
    \item Probar la capacidad del sistema para concentrar ion litio
    \item Adaptar el método desarrollado para extraer y concentrar selectivamente ion litio presente en agua de mar.
    \item Programar un paquete de \verb|R| que permita automatizar tanto como sea posible el tratamiento de datos, para la producción de resultados numéricos y gráficos de una manera sencilla, consistente, y reproducible.
\end{itemize}

\section{Hipótesis}
Pueden encontrarse condiciones que permitan extraer y concentrar selectivamente ion litio presente en agua de mar usando membranas poliméricas de inclusión de triacetato de celulosa. Dicha membrana debe contener en su formulación, extractantes como los que han sido previamente reportados para el recobro de ion litio en extracciones sinérgicas con disolventes o usando membranas liquidas soportadas. Los diseños de experimentos y algoritmos de optimización pueden ayudar en el proceso minimizando el número de experimentos requeridos para el fin propuesto.


\section{Estructura del documento}
Esta tesis se encuentra dividida en siete capítulos, de los cuales en el primero se ha descrito el problema que atañe al presente trabajo y el enfoque desde el cual se ha decidido abordar. Tras leer este capítulo se espera que el lector cuente con la información suficiente para decidir si el contenido del presente documento es o no de su interés, con el propósito ideal de motivarlo a seguir con las demás partes del escrito o bien, para no hacerle perder más de su valioso tiempo.

En el segundo capítulo se pone en contexto el trabajo realizado, iniciando con una descripción del litio como un elemento estratégico, transversal a distintos sectores económicos, y de gran importancia para diversas tecnologías que ganan cada vez más protagonismo en la sociedad. Los aspectos geográficos y económicos de los recursos mundiales de litio son analizados brevemente. Se da un panorama general de las técnicas aplicadas a escala industrial para su recobro y se mencionan las técnicas que han sido recientemente desarrolladas por grupos de investigación esparcidos en todo el mundo para este fin. En este capítulo se introduce el concepto de membranas poliméricas de inclusión que, como se mencionó en el presente capítulo, fueron las escogidas para abordar el problema planteado. Finalmente, se hace un breve recuento sobre algunos conceptos fundamentales de diseño de experimentos, un conjunto de metodologías que constituyeron una de las piedras angulares para el éxito del presente proyecto. 

El tercer capítulo propone una ecuación empírica para modelar los perfiles de transporte. Este modelo presenta una alternativa similar al que se planteó en nuestro grupo de investigación hace unos años \citep{RODRIGUEZDESANMIGUEL2014}. Las ecuaciones empíricas son útiles en la optimización de sistemas de transporte haciendo uso de los diseños de experimentos, gracias a que sus parámetros ajustables pueden servir para calificar los resultados de un experimento particular.

El desarrollo experimental seguido durante el proyecto de maestría se ilustra en el cuarto capítulo. Se intentó incluir toda la información necesaria para repetir los experimentos realizados en aras de replicar los resultados obtenidos si así se desea. Los detalles de composición de las membranas y las disoluciones empleadas se decidieron a medida que avanzó la optimización del sistema por lo que estos se muestran en el capítulo de resultados y discusión de resultados. El tercer capítulo se complementa con los Anexos \ref{sec:quantification}, \ref{App:tracker}, y \ref{Sec:microfuck}, que describen protocolos experimentales que no resultan imprescindibles para entender los resultados, pero que sí fueron fundamentales para la obtención de los mismos.

El quinto capítulo habla del paquete \verb|transmem|, que permite obtener parámetros de desempeño de los sistemas y producir representaciones gráficas de alta calidad. Este capítulo se complementa con el Anexo \ref{sec:transmemManual} que corresponde al manual oficial del paquete. Dicho manual describe todas las funciones del paquete y su uso apropiado, mediante ejemplos prácticos que usan bases de datos producidas en el desarrollo de esta tesis y que fueron incluidas en el paquete de \R. 

El sexto capítulo contiene los resultados y la discusión de los mismos. Este capítulo compone el corazón del trabajo de tesis presentado. Se ha buscado evidenciar la lógica bajo la cual fueron tomadas las distintas decisiones que desembocaron en el producto final. Por facilidad, para hacer alusión a algunos de los distintos sistemas ensayados, a cada membrana se le ha asignado un único identificador compuesto por una letra y un número separados por un punto. La letra indica la serie de experimentos a la que pertenece la membrana y el número indica el orden de elaboración o de uso dentro de la misma serie. 

Las conclusiones obtenidas a partir de los resultados presentados están en el séptimo capítulo. Se hace alusión a los objetivos presentados al comienzo del documento, y se discute si dichos objetivos han sido alcanzados. Se incluyen algunas perspectivas que podrían direccionar algún trabajo futuro que desee complementar el aquí presentado. 

Los anexos contienen (en orden) un artículo aceptado por la revista \textit{Desalination} de la editorial neerlandés Elsevier, los detalles de la cuantificación instrumental de cationes en disolución, la metodología para la determinación de la velocidad de giro en las propelas de las celdas de permeación, la metodología de microtitulación gravimétrica ácido-base, y el manual del usuario para el paquete \verb|transmem|.
\clearpage
\ChapBib{chap1/introduction}
